\documentclass[10pt, oneside]{article} 

\title{Outline (25 pages)}
\author{[Full Name]}
\date{\today}

\begin{document}

\maketitle
%\tableofcontents

\vspace{.25in}


\section{Introduction--5 pages}
\subsection {Background and Context 14 hours}
\begin{itemize}
\item Affordable Care Act and Preventive Services
\item What is cost sharing
\item Why Preventive Care is important
\item Benefits of Preventive Services
\item Disparities in Preventive Care 
\item Cancer Screening among Immigrants
\item preventive care recommendation

\end{itemize}


\subsection{Research Gap Identification: 5 hours}
\begin{itemize}
\item Highlighting the limited attention given to the impact of ACA and Medicaid Expansion on foreign-born 
\end{itemize}


\subsection{Literature Review, 18 hours}


\begin{itemize}
\item What are studies on ACA and preventive care? 
\item What are the studies on cancer screening and ACA
\item Mention key findings and insights from previous research.
\item Pointing out the limitations of previous research
\item Explain why new research is needed to address these limitations.
\item Describing the challenges faced by foreign-born individuals in accessing healthcare.
\end{itemize}

\subsection{purpose statements}
\begin{itemize}
\item Emphasizing the significance of evaluating the impact of Medicaid Expansion on healthcare disparities.
\item The need for tailored interventions and improved healthcare equity.

\end{itemize}


\section{Theoretical Frameworks--5 pages( 4 hours)}

\begin{itemize}
\item explaining the disparity among Foreign-Born and US-Born
    \item Discussing the health service utilization models?
    \item Intersectionality 
    \item Empirical Research Barriers and Predictors to receiving Preventive Services

\end{itemize}






\subsection{Research Objectives and Contributions}

\begin{itemize}
\item Presenting the specific research questions that the study aims to address
\item List the research questions that your study seeks to answer.
\item Frame these questions within the context of ACA and Medicaid Expansion's impact on foreign-born individuals.
\item Highlighting the innovative methodology and approach used in the study
\item Describe the unique methodology and approach that your study employs to address the research questions.
\item Explain how this approach differs from previous research methods.
\item Mentioning the unique contributions the study aims to make to the existing literature
\item Identify the novel contributions and insights that your study intends to provide to the field of healthcare research.
\item Explain why these contributions are valuable and relevant.

\end{itemize}


\section{Methodology 5 pages}

\subsection{Data}
\begin{itemize}

\item Main Source: Medical Expenditure Panel Survey Household Component MEPS-HC
\item Data Collection Process: 
\begin{itemize}
    \item Selects sample households annually from previous NHIS participants, conducting interviews for two years with five data collection rounds.
    \item Gathers extensive information on demographics, socioeconomic status, health insurance, healthcare access, health status, conditions, utilization, and expenditures.
    \item Medical conditions classified by expert coders with AHRQ-verified error rate below 2.5\%.
\end{itemize}
\end{itemize}

\subsection{Sample}

\begin{itemize}
    \item Include nine years of cross-sectional data (2011-2019)
\end{itemize}

\subsection{Variables}


\subsubsection{Dependent variables}
\subsubsection{Key independent variables}
\subsubsection{Other independent variables}
\subsection{Empirical Approach}

\section{Results --5 pages}

\section{Discussion --5 pages}




\end{document}



