\documentclass[../main.tex]{subfiles}

\begin{document}

In this study, I developed my conceptual framework with a particular focus on the latest iteration of Anderson's Behavioral Model of Health Services Use (BMHSU). I chose this model over other health care access modeles such as Penchansky’s Model and The Institute of Medicine (IOM) Model of Access Monitoring \cite{karikari-martin_use_2010}, due to its comprehensiveness and compatibility with my research objectives and its frequent citation and application in studies related to the ACA.

Since its development by Anderson in the 1960s, the BMHSU model has undergone significant revisions in response to advancements in healthcare research.

Initially, its emphasis was on understanding the reasons behind families' healthcare service utilization, highlighting predisposing characteristics, enabling factors, and the need for care. Subsequent phases of the model introduced enhancements to better capture healthcare dynamics. Phase 2 integrated the healthcare system and consumer satisfaction as outcomes. Phase 3 acknowledged personal health practices and the healthcare system's goal to maintain and improve health, incorporating health status as an outcome. This phase also extended access measures for health policy and reform. Phase 4 introduced the dynamic nature of the model, illustrating multiple determinants of healthcare use and health status, including feedback loops between outcomes and characteristics. Phase 5 emphasized understanding healthcare use through contextual and individual determinants. It divided contextual characteristics into predisposing, enabling, and need components, similar to individual characteristics. Additionally, it incorporated the process of medical care as a type of health behavior within the model, stressing the importance of provider-patient interactions. In its most recent update, Phase 6 of the model saw further expansion, incorporating genetic susceptibility as a new individual predisposing factor and introducing quality of life as an additional outcome measure. See fig1 for the last revision of model

While this model is perfectly explaining the health care use of general population when in comes to vulnerable population, the models needs further modifications to be used. 


To address the specific needs of foreign-born populations, the standard framework was adapted to incorporate additional relevant factors. A more focused and analytical framework was developed to guide the research questions of the study, ensuring a comprehensive examination of healthcare access within this demographic.



\end{document}

Predisposing contextual characteristics encompass demographic and social factors, as well as community beliefs regarding healthcare. Enabling contextual characteristics include health policies, financing mechanisms, and organizational structures that facilitate or hinder access to healthcare. Need contextual characteristics involve environmental and population health indicators that influence the perceived and evaluated need for medical care.

Similarly, individual characteristics are divided into predisposing, enabling, and need categories. Predisposing individual characteristics encompass demographic and social factors, such as age, sex, education, occupation, and health beliefs. Enabling individual characteristics include financial resources, social support networks, and access to healthcare facilities. Need individual characteristics encompass both perceived and evaluated need, influenced by individuals' health perceptions and professional assessments of their medical requirements.

Health behaviors, which encompass lifestyle practices like diet, exercise, stress management, and adherence to medical regimens, are integral to understanding individual interactions with healthcare. The process of medical care involves various aspects of provider-patient interactions, including counseling, test ordering, and communication. Utilization of personal health services is influenced by predisposition, resources, and perceived need.

Outcome measures in healthcare access include perceived and evaluated health statuses, consumer satisfaction with care, and overall quality of life. Quality of life extends across physical, psychological, social, and environmental domains, reflecting the holistic nature of well-being.