\documentclass[../main.tex]{subfiles}

\begin{document}

\section{Data Source}

\subsection{kelchi thesis}
Because the study seeks to explore repeated measures or panel data from 2011–
2019, a longitudinal quantitative analysis of the Medical Expenditure Panel Survey
(MEPS) (a large, publicly available data set) will be employed (Cohen et al., 2009; Gai \&
Jones, 2020; Young et al., 2020). The Department of Health and Human Services
(DHHS) conducted nationally representative population-based surveys of health care
expenditures and utilization in 1977 and 1987 (Waldo et al., 1989). MEPS was initiated
in 1996 to provide more timely information by adding a new panel or sample of
households into the survey every year rather than every 10 years (AHRQ, 2014, 2022b;
Cohen et al., 2009). MEPS provides annual estimates of health care utilization, insurance 
40
coverage, expenditures, and sources of payment for the U.S. civilian non-institutionalized
population (individual and household) (AHRQ, 2014, 2022b; Cohen et al., 2009).
Cosponsored by two DHHS agencies, the Agency for Healthcare Research and Quality
(AHRQ), the Centers for Disease Control and Prevention (CDC), and the National Center
for Health Statistics (NCHS), the MEPS questionnaire consists of three interrelated
surveys: the Household Component (MEPS-HC), the Medical Component (MEPS-MC),
and the Insurance Component (MEPS-IC) (AHRQ, 2014, 2022b; Cohen et al., 2009).
The MEPS oversamples the following policy-relevant populations groups: Blacks,
Hispanics, Asians, and low-income households to improve the precision of household
estimates (Blewett et al., 2016; Cohen et al., 2009); and consists of an overlapping panel
design in which the same individuals and households are interviewed a total of five times
over a course of 30 months – yielding data for two calendar years (repeated measures)
(Cohen et al., 2009). Aday and Andersen argue that a household survey is the best
method for collecting data on at-risk populations (Aday \& Andersen, 1974).
Consequently, the MEPS was identified as an ideal data set because of its multi-year
coverage, socio-demographic diversity, and large sample size (Cohen et al., 2009; Cohen
\& Cohen, 2013).
Additionally, this survey provides analysts with policy-relevant information
across a variety of contexts; and has been used extensively in analyzing the impacts of
the ACA (e.g.., health care expenses, utilization, insurance coverage, and access to care
etc.) (Cohen et al., 2009; Cohen \& Cohen, 2013). Following are the ways in which the
MEPS defines insurance status: 1) any private insurance describes any individual who, at
any time during the year, had insurance that provides coverage for hospital and physician 
41
care 2) any public insurance describes any individual covered by Medicare,
Medicaid/CHIP, or other public hospital/physician coverage at any time during the year
and 3) uninsured describes anyone not covered by private hospital/physician insurance at
any time during the survey year (Davis, 2001).
While MEPS provides for different gradients of insurance analysis over the
course of months within a given year (i.e., having private or public insurance for 12
consecutive months, insured for part of the year, or insured for at least one month), part
of the aim of this study was to examine for any insurance coverage. Thus, a data
limitation is the potential diluting effect of evaluating insurance coverage in the
aggregate. Future studies may consider further disaggregating the MEPS insurance
gradient based on the researcher agenda.
The data for this study will be drawn from the household component of the survey
because it aligns with the study questions and has relevant variables to assess specific
components of the Aday-Andersen Access to Care framework (e.g., demographics, predisposing factors, enabling factors, need factors, health conditions, usual source of care,
employment, income, insurance type, access to care etc.). Regarding power and sample
size considerations, there has been some debate (Dziak et al., 2020) on the utility of
conducting power and sample size calculations for analysis of data that have already been
collected - as is the case with the MEPS data. Given that context, an exploratory analysis
was conducted to determine the specific number of non-Hispanic Black and non-Hispanic
White men who contributed at least one observation to the MEPS database between 2011
and 2019 (see Figure 4 and Figure 5 in the appendix). During that timeframe, the number
of non-Hispanic Black men contributing at least one observation ranges from a low of 
42
1001 (in 2019) to a high of 2024 (in 2012); and the smallest number of non-Hispanic
White men contributing at least one observation ranges from a low of 3584 (in 2014) to a
high of 4149 (in 2011 and 2012) (see Table 2 below) (AHRQ, 2022b).

\subsection{Wang paper}
MEPS is a subsample of respondents to NHIS. Five MEPS data components were merged for this purpose, including full-year consolidated data files, emergency department visits files, office-based medical provider visits files, outpatient visits files, and hospital inpatient stays files. Conducted by the Agency for Healthcare Research and Quality (AHRQ), MEPS is an ongoing annual survey collecting high quality micro-level data through household and medical provider interviews, including demographic characteristics, socioeconomic status, health status, health services utilization and expenditures. Medical conditions associated with each visit were coded by professional coders by through verbatim text reported by each respondent. The codes have been verified by AHRQ to have an error rate less than 2.5\%. The survey data are publicly available and nationally representative of the non-institutionalized US population. The linkage between MEPS and NHIS provided individual place of birth and current citizenship, which were used to define immigrant status in the study. Our original sample contained 113,499 adults aged 18 and older. After excluding respondents with any missing values in covariates (3.4\% of the sample), the final analytical sample size was 109,602, including 80,911 US natives, 12,809 naturalized citizens, and 15,882 noncitizens.

\subsection{Bloodworth thesis}

edical Expenditures Panel Survey (MEPS). My study used four data sets to test the study objectives. First is individual-level data from the Medical Expenditure Panel Survey (MEPS). Beginning in 1996, MEPS is a set of large scale surveys that assesses healthcare services utilized by Americans (Agency for Healthcare Research \& Quality (AHRQ), 2009). MEPS includes detailed demographic information, as well as information on health conditions, health status, use of medical services, charges and source of payments, access to care, satisfaction with care, health insurance coverage, income, and employment (AHRQ, 2009). From MEPS, I used data on preventive service utilization as well as demographic, socioeconomic, geographic, and health status and access variables. The MEPS data were linked with Kaiser data on Medicaid preventive service policies and expansion status at the state level, described in more detail below.
\subsection{Xiaobei thesis }

The primary data source for the study was the Medical Expenditure Panel Survey – Household Component (MEPS-HC) administered by the Agency for Healthcare Research and Quality (AHRQ). MEPS-HC is a nationally representative survey of the U. S. civilian noninstitutionalized population. It collects data from households and their members on a wide range of topics including health insurance coverage, access to care, health service utilization and expenditures, health status, and demographic characteristics. MEPS-HC’s sampling frame is drawn from respondents to the prior year’s National Health Interview Survey conducted by the National Center for Health Statistics. Each year, MEPS-HC selects a new panel of sample households and collects data from each panel for two calendar years in five rounds of interviews. Through an overlapping panel design, each calendar year is covered by data from two panels. Panels for years 2007 to 2017 were pooled to create a repeated cross-sectional dataset that covered time periods both before and after the Medicaid expansion. All data are publicly available except for the state and county identifier variables, both of which were needed to link with MEPS the state- and county-level external data detailed in the following section. In addition, the state identifier variable was needed to construct the independent variable of interest, which identified whether a respondent resided in a state that expanded its Medicaid program, and to specify state fixed-effects in the Difference-in-Differences regression models. Both identifier variables were obtained through an application to AHRQ.


Other Data Sources Data on three control variables came from other sources. The state-level variable, annual unemployment rate, was drawn from the publicly available Local Area Unemployment Statistics provided by the U. S. Bureau of Labor Statistics (BLS). Two county-level control variables, urbanicity and the ratio of physicians per 1,000 people, were derived from the Area Health Resource Files (AHRF) maintained by the Health Resources and Services Administration (HRSA). The BLS data were provided to AHRQ in the same application for the agency to merge with MEPS. The AHRF data were directly obtained by AHRQ from HRSA and then also merged with MEPS. Initially, the ratio of hospital beds per capita was intended to be included as a county-level control variable. However, upon gaining access to the data, I noted an unusually large number of zero values on the hospital bed variable. Further investigation and communication with AHRF revealed that missing data were treated as zero prior to 2013-2014. Because there was no way to differentiate true zeroes from missing values, the variable was not used in the study.

\subsection{Holden Carig David Thesis}
Medical Expenditure Panel Survey (MEPS) data were used for this study. MEPS is a household survey of the civilian non-institutionalized population of the U.S sponsored by the Agency for Healthcare Research and Quality and the National Center for Health Statistics and provides representative estimates of healthcare utilization, insurance coverage, and sociodemographic characteristics of the population (Ezatti-Rice, Rohde, \& Greenblatt, 2008). Eight years of cross-sectional household component MEPS data were pooled (2004 through 2011); analysis took place in 2013. The analytic sample included individuals aged older than 18 years without health insurance for the previous year and was limited to non-Hispanic whites, non-Hispanic African Americans, and Hispanics in order to have sufficient statistical power to compare preventive service utilization across race and ethnicity. This study employed USPSTF recommendations for preventive services to assess proportion and odds of receipt by race/ethnicity and income.

Data for this study were obtained from the Medical Expenditure Panel Survey (MEPS), a household survey of the civilian non-institutionalized population of the U.S. sponsored by the Agency for Healthcare Research and Quality and the National Center for Health Statistics. The sampling scheme and methods for data collection in the MEPS have been described in detail (Ezatti-Rice et al., 2008). MEPS provides representative estimates of health care utilization, insurance coverage, and sociodemographic characteristics of the population. Six years of cross-sectional household component MEPS data were pooled and grouped into pre-period (2007 to 2010) and post-period (2011 to 2012) in relation to the implementation of the ACA. The sample was restricted to non-Hispanic Whites (Whites), non-Hispanic African Americans (African Americans), and Hispanics in order to have sufficient statistical power to compare preventive service utilization across race and ethnicity. The analyses were further restricted to those aged 65–75 years to include Medicare enrollees likely to benefıt most from CRC screening.

\subsection{Kindratt thesis}
Since 1996, the MEPS has collected information on sociodemographic factors, health care utilization, expenditures and health insurance coverage from nationally representative samples of adults in the US using a survey panel design (17). From 2011-2015, households recruited for each panel (household component) were selected based on a subsample of households who participated in the previous year’s NHIS. The panel design included five face-to-face interviews conducted over a two-year span using a computer-assisted personal interview (CAPI) system and self-administered questionnaires (17). Cancer screening questions were collected using the CAPI system and face-to-face PPC measures were collected using self-administered questionnaires. After completion of the household interviews, medical providers were contacted by telephone to provide additional details on medical visit summaries, diagnostic codes, and billing (17). Further details of the MEPS sampling design and data collection methods have been reported previously (46).

Inclusion:
The sample was limited to women ages 40 and older who reported a primary health care visit in the last 12 months with no personal history of breast cancer based on ACOG screening recommendations (41). The total sample consisted of 69,749 women (MEPS n=14,887; HINTS n=4,773, NHIS n=50,089) which when weighted, represented 311,705,232 women.


\subsection{Tossef}

We conduct our analysis using repeated cross-sectional data from the household component (HC) files of the Medical Expenditure Panel Survey (MEPS), which spans 11 years and covers five years of pre-ACA (2006–2010) and five years of post-ACA (2012–2016) reforms. MEPS is sponsored and conducted by the Agency for Healthcare Research and Quality (AHRQ), and provides nationally generalizable estimates of healthcare access, utilization, and expenditures among noninstitutionalized residents of the U.S. Details regarding the MEPS, its component files, structure, design, and data modules are available elsewhere

Inclusion: 
Analyses are restricted to individuals 65 years or older with Medicare insurance. The full sample of individuals in MEPS 2006–2016 includes N = 384,060 individuals. After excluding the year 2011 and including in-scope individuals (Whites, Hispanics and Blacks) with Medicare and checking for missing values in all the covariates, our analytic subpopulation consists of 27,124 Medicare beneficiaries (65 years or older). Detailed specifications of participant inclusion and unweighted Ns are provided in Supplemental Fig. 1.


\subsection{ another Tossef 2019}
Data are drawn from the 2003–2012 MEPS, which is nationally representative of the non-institutionalized US population. For these years, MEPS Household Component (HC) files were merged with the MEPS Full-Year Consolidated and Hospital Inpatient Stays files to create a file of all MEPS respondents with hospital stays. This sample was then restricted to individuals with Medicaid coverage, ages 18–64, and this set was further subdivided into duals and non-duals. After excluding respondents with any missing values in the covariates, the final number of duals and non-duals in our analytic sample was 515 and 2937, respectively. Data for 2013 and beyond are excluded from this study because after 2012 MEPS does not include the ICD-9 codes in the publicly available files, and we use ICD-9 codes to identify preventable hospitalizations. Additionally, since many Medicaid programs in 2013 and 2014 raised their payment rates to primary care physicians to no <100\% of Medicare payment rates for primary care services (an Affordable Care Act provision), we stopped at 2012 to enhance the precision of our estimates.

The MEPS is based on a complex survey design that involves stratification, clustering and disproportionate sampling (AHRQ, 2014b). Our models and estimates account for these design elements, and our analyses were conducted using survey command functionalities in Stata v.13 (StataCorp, 2013).



\subsection{Gawron}
MEPS is a set of publicly available large-scale surveys of families and individuals, medical providers, and employers in the USA [9]. All data are publicly available and deidentified; IRB approval is not necessary for use of MEPS data. We used the Consolidated Household Component data file of the publicly available MEPS for the years 2000 through 2016 [10]. The survey included questions related to screening tests and addressed stool testing for occult blood or endoscopic assessment with colonoscopies or flexible sigmoidoscopies without differentiating between these two procedures. Published response rates of the surveys ranged from 71.4 to 51.0\% (mean 62.5 ± 6.3\%) during the time period analyzed. Considering the focus on CRC screening, we excluded persons under the age of 50 years.

\section{variables}

\cite{courtemanche_revisiting_2023}

codes for prevetable death form CDC compresed death file
Following Nolte and McKee (2012), healthcare amenable mortality is defined as deaths potentially preventable given effective and timely health care from the following causes of death as classified by the International Classification of Diseases (10th revision): infectious diseases (A00-9, A15-9, A35, A36, A37, A80, B05, B90), tumors (C18-21, C44, C50, C53, C54, C55, C62, C81, C91-5), diabetes (E10-4), heart (I20-5, I05-9, I10-3, I15, I609) and respiratory (J00-9, J20-99, J10-1, J12-8) diseases as well as surgical (K25-7, K35-8, K40-6, K80-1, N007,N17-9, N25-7, N40), maternal (O00-99, Q20-8), pre-natal (P00-96, A33) or other (E00-7, G40-1) conditions.



\subsection{}
Data for this study will be extracted from the MEPS using cleaned full-year
2011–2019 consolidation files (AHRQ, 2019). Specifically, two access to care indicators
will be operationalized: 1) predisposing factors (e.g., race/ethnicity, age, education) and
2) enabling factors (insurance coverage and usual source of care) (Aday \& Andersen,
1974; Andersen \& Davidson, 2007; Luft et al., 1976). Other variables of interest include:
age, education level, employment status, income level, and region of the country
(Northeast, Midwest, South, and West) – to be described in more detail below.
The specific variables identified from MEPS were as follows: RACETHX was
used to represent race, POVCAT was used to represent income, EMPST31 was used to
represent employment, REGION1X was used to represent region, INSCOV1X was used
to represent insurance status, and HAVEUS42 was used to represent usual source of care.
Regarding employment, this variable includes categories for “job to return to” and “job
during” the survey interview. Because part of the focus of this study is on whether nonHispanic Black men and non-Hispanic White men are employed, the previously
mentioned categories will be recoded and labeled as “Intermittent Employment” in order
to preserve the totality of the observations in the data. Specifically, MEPS defines fulltime employment as working at least 35 hours per week; whereas, part-time employment
is defined as working less than 35 hours per week (AHRQ, 2022b; Carroll, 2018).
Because part-time workers typically do not have equal access to comprehensive benefits
compared to full-time (year-round workers), this study will explore how these differences
impact access to care (Brault \& US, 2011; Su et al., 2019).

Regarding insurance status, while MEPS provides for different gradients of
insurance analysis over the course of months within a given year (i.e., having private or
public insurance for 12 consecutive months, insured for part of the year, or insured for at
least one month), part of the aim of this study was to examine for any insurance coverage.
Thus, a noted data limitation is the potential diluting effect of evaluating insurance
coverage in the aggregate. However, this study is unique in that it is the first of its kind to
specifically assess access to care in a granular way between non-Hispanic Black men and
non-Hispanic White men pre- and post-ACA.
As the variable names for education level (EDUCYR) were not consistent across
the years included in the sample, education was recoded into the following categories
based on the variables available in each year: 1 = no degree, 2 = GED or high school
degree, 3 = some college or two-year degree, 4 = bachelor’s degree, 5 = master’s,
doctoral or professional degree, 6 = other degree. In order to preserve the totality of the
data observations, category number three (some college or two-year degree) will be
combined with category number four (bachelor’s degree). Likewise, category number six
(other degree) will be combined with category number five (master’s, doctoral, or
professional degree). Age (AGELAST) was recoded into a categorical variable with the
following categories: 1 = 18 to 34, 2 = 35 to 49, 3 = 50 to 64. Time was coded as a
categorical variable with the following categories: 0 = pre-ACA (2011 to 2014), 1 = postACA (2015 to 2019).
Dependent or Outcome Variables
In order to assess study questions 1.1–1.3, insurance status (e.g.., insured and
uninsured) and usual source of care (i.e., yes, no) will be examined (Blewett et al., 2008; 
45
Chen et al., 2016; Hammond et al., 2011; Stewart et al., 2019; Xu, 2002). The insured
category will include any private and any public insurance. Insurance status will be
specified as model #1 (see Table 3 below). The second key access to care indicator
variable, usual source of care, ascertains whether there is a particular doctor’s office,
clinic, health center, or other place that the individual usually goes to if he/she is sick or
needs advice about his/her health (AHRQ, 2019). This study will construct a
dichotomous (yes/no) variable indicating whether the person’s provider is in an office;
but not a hospital nor an emergency room.
There are three main levels of care in a hospital: inpatient, outpatient, and
emergency. Inpatient care is for patients who need to stay in the hospital for more than 24
hours. Outpatient care is for patients who can go home after their treatment. Emergency
care is for patients who need immediate medical attention (Bovonratwet et al., 2017; Liu
\& Kelz, 2018). Further, studies have characterized the emergency room as outside of the
usual source of care category due to its acute, transitory structure (Janke et al., 2015).
Therefore, the emergency room component will not be categorized as being a usual
source of care. However, usual source of care will be specified as the response (yes/no)
for model #2 (see Table 3 below). Noted previously, having a usual source of care
potentially increases timely receipt of preventive screenings (Powell et al., 2020) and has
been implicated in previous studies examining the health of Black men and other
segments of the population (Blewett et al., 2008; Hammond et al., 2011; Stewart et al.,
2019; Wallace et al., 2022).
46
Primary Independent or Predictor Variables
The independent or predictor variables in this study will be race/ethnicity and
time. The two time periods are 2011–2014 (pre-ACA) and 2015–2019 (post-ACA). Race
and ethnicity will be classified as non-Hispanic Black and non-Hispanic White to assess
the differences in outcomes by race. Hispanics or Latinos may be of any race (Aragones
et al., 2014), and the U.S. Census Bureau and the Office of Management and Budget
(OMB) reports Hispanics separately from non-Hispanics for each race group (OMB,
1997; United States Census, 2022); thus, this classification approach will be followed for
this study (OMB, 1997; United States Census, 2022). The time variable will be
operationalized as pre/post-ACA; and analyzed as a binary variable in order to estimate
changes over time across study subjects in the database.
Covariates
Noted earlier, the Andersen Model of Health Care Utilization suggests the
following factors are important predictors of access to care: 1) predisposing or
demographic factors (e.g., race/ethnicity, age, education) and 2) enabling factors (e.g.,
insurance coverage, usual source of care, employment, income, region) (Aday \&
Andersen, 1974; Axon et al., 2022; Barker et al., 2021; Li et al., 2016). This list of
potential covariates will therefore be operationalized for this study from the MEPS
dataset to specifically include age, education level, employment status, income level, and
region of the country (Northeast, Midwest, South, and West).
First, policy changes impact people differently by age (Sohn, 2017). Historically,
young adults have been at the greatest risk for being uninsured and less likely to have a
usual source of care compared to older adults (Monaghan, 2014). In a 2011 study, young 
47
adult males were the least likely to have health insurance coverage, with 28.3 % of males
ages 18–24 reporting no insurance coverage; and also found that only 77.9 % of women
and 62.5 % of men aged 19–25 years reported a usual place for health care (Cohen,
2019). Data from the 2009 Medical Expenditure Panel Survey also found that young
adults ages 18–26 had the lowest health utilization rate of any age group (Lau et al.,
2014). Therefore, one of the sections (Dependent Coverage Mandate) of the ACA
provides for children to remain on their parent’s health insurance coverage up until age
26 (DHHS, 2022; Look \& Arora, 2016). Noted earlier, this study seeks to focus on
working age adults (18–64) because individuals in this age range are more likely to gain
or have insurance coverage through the ACA (Angier et al., 2020). Therefore, this study
stratifies age into three categories (18–34, 35–49, and 50–64) to explore how the ACA
impacted access to care across age groups.
Many studies have found employment status, income, and region of the country to
be significant determinants of health and access to care – which motivated their inclusion
as covariates (Braveman \& Gottlieb, 2014; Bruce et al., 2015; Buchmueller \& Levy,
2020; Grogan \& Park, 2017). Studies have found complex relationships between
employment status and insurance status (Reichard et al., 2019). For example, one study
found that those who worked full-time are more likely to report poor health outcomes
than part-time workers, regardless of the type of employment (i.e., small employers, full
and part time permanent employees, etc.) (Benavides et al., 2000).
While income is a critical determinant of health (Abdalla et al., 2020; Assari,
2018; Braveman \& Gottlieb, 2014; Dubay \& Lebrun, 2012; Thorpe et al., 2013; Wilson
et al., 2017), respondents of color and other marginalized communities tend to skip 
48
income-related questions in surveys (either because of lack of trust or the wording of the
survey questions) (Pinto et al., 2022); however, it will be evaluated in this study. Lastly,
region of the country is important to assess in order to build the evidence case for further
expanding health insurance coverage and determining more sustainable interventions in
disproportionately poor health outcome regions of the country like the South and
Southeast (Howard, 1999; Obisesan et al., 2000; Perry \& Roccella, 1998). 

\subsection{Bloodworth thesis}
Health status variables included number of chronic conditions (no chronic conditions* versus one chronic condition and multiple chronic conditions) and self-reported health status (poor, fair, or good health* versus very good or excellent health).

\subsection{Hong et al.}
English proficiency and English use of family members were assessed by asking whether a respondent/family member was comfortable conversing in English. Responses were coded as 1 (no)or2(yes). Language preference was measured according to which language was spoken at home and was coded as 1 (other languages)or2 (English). The internal reliability was good (a 5 .82).
Health needs The number of comorbid conditions was used to measure health needs. Comorbidity was defined as a “yes” response from 5 yes-or-no items about health conditions (hypertension, congestive heart failure, heart disease, high cholesterol, and diabetes mellitus). The number of comorbidities was recoded as an ordinal score of 0, 1, 2, or more. Perceived risk of health status Perceived risk of health status was assessed according to perceived physical and mental health status. Possible choices ranged from 1 (excellent)to5(poor). A composite score was computed by determining the arithmetic mean. The 2 scales had good internal consistency (a 5 .83).
Full-year insurance status was used and recoded as 1 if an individual was insured by any type of insurance (any private or public) and as 0 if an individual was uninsured. Access to care Access to care was assessed with 3 items. Respondents were asked to indicate how often 1) they made an appointment for health care as soon as needed; 2) they received care as soon as needed; and 3) it was easy to get the care, tests, or treatments that were necessary. Response scales ranged from 1 (never)to4(always). Mean scores were calculated, with higher scores indicating better access to care. The internal reliability was excellent (a 5 .88). Perceived quality of health services Perceived quality of health services was assessed using 5 items about patient experience with receiving health care and scores of overall satisfaction. Respondents were asked to indicate how often health providers: 1) listened carefully, 2) explained things in a manner that was easy to understand, 3) showed respect, 4) spent enough time, and 5) provided advice/instruction that was easy to understand. Response scales ranged from 1 (never)to4(always).

Overall satisfaction was measured by rating health care from all health providers on a scale from 0 (worst health care possible)to10(best health care possible). We calculated a score by weighting equally and averaging the 6 responses, in which a higher score indicated higher perceived quality of health services. The internal consistency of the total score was excellent (a 5 .90). Perceived distrust in health care Perceived distrust in health care was measured with 4 items: 1) no need of health insurance, 2) not worth having insurance, 3) more likely to take risks than the average person, and 4) able to overcome without help from health providers. The 5-point scale ranged from 1 (disagree strongly)to5(agree strongly). The average score was computed, with higher scores representing higher distrust in health care. The internal consistency reliability was acceptable (a 5 .70).

\subsection{Xiaobei thesis }
Following Miller, Kirk, Kaiser, and Glos (2014), I created an insurance variable with five mutually exclusive categories: Medicaid, Medicare, dual (Medicaid/Medicare), private insurance, and uninsured. The categories were constructed based on the MEPS variables defined as follows: For the first four types of categories, a value of 1 indicated that the respondent was covered by the subject type of insurance for at least one day during the survey year. A value of 0 indicated that the person was not covered for all of the survey year. For the uninsured category, a value of 1 indicated that the person was

Hospital emergency room visits. A dummy variable was created with a value of 1 indicating that a respondent made at least one visit for the survey year. A value of 0 indicated that no visit was made.

Office-based physician visits. A dummy variable was created with a value of 1 indicating that a respondent made at least one visit for the survey year, and a value of 0 indicating that no visit was made.

Predisposing: • Age • Sex • Race/Ethnicity • Marital status • Family size • Educational attainment • Employment status

Enabling • Family income as a percentage of the Federal Poverty Line based on family size and composition

Need • Disability: Basic, Complex, and Both (as discussed above) • General health status

Insurance coverage • A five-category variable: Medicaid, Medicare, Dual (Medicaid/Medicare), Private, Uninsured • MCDEV, MCREV, PRVEV, UNINS Access to care • Having a USC • Unmet medical care needs • Unmet prescription medicine needs • HAVEUS, PROVTY • MDUNAB, MDDLAY • PMUNAB, PMDLAY Health service use • Hospital emergency room visits • Office-based physician visits • ERTOT • OBDRV Health outcome • Number of days missed from work due to health problems • DDNWRK



\subsection{Carig David Thesis}

Dependent Variables 
Eight preventive services were analyzed, including cancer screening procedures (breast, colon, and cervical), hypertension and blood cholesterol screening, routine physical checkup, receiving doctor’s advice to quit smoking, and influenza vaccination. Following USPSTF guidelines, eight binary indicator variables were created to reflect whether individuals met guidelines. Table 1 summarizes the analyzed preventive services by recommended population, frequency of assessment, MEPS survey question(s), and resultant sample sizes.


Table 1. U.S. Preventive Services Task Force preventive service measures Screening Recommended Population Frequency Survey Question(s) Uninsured 
Sample
\begin{itemize}
    \item Cervical Cancer (Pap Smear) Women aged 2165 Every 3 years How long since last Pap smear test? 12,192 Had a hysterectomy? 
    \item Breast Cancer (Mammogram) Women aged 4074 Every 2 years How long since last mammogram? 6,764 
    \item Colorectal Cancer Adults aged 50-75 Based on screening: 2004-2008: When was last sigmoidoscopy? 6,355 Fecal Occult Test Fecal Occult yearly, or When was last blood stool test?
    Colonoscopy Colonoscopy every 10 years, or 2009-2011: When was last blood stool test? Sigmoidoscopy Sigmoidoscopy every 5 years When was last colonoscopy? with fecal occult every 3 years. When was last sigmoidoscopy?
    
\end{itemize}
other variables:
race/ethnicity and household income. Race/ethnicity was categorized as white non-Hispanic, African American non-Hispanic, and Hispanic. Categorical variables for four poverty groups were assigned corresponding to poor, low, middle, and high income according to the Federal Poverty Level (FPL): <100\%, 100\%–199\%, 200\%–399\%, and ≥400\% FPL, respectively.
Marital status was categorized as never married, married, and divorced/separated/widowed. Education was categorized as less than high school, high school, some college, and college graduate. Primary language, based on interview language, was categorized as English or other. Age categories were: 18–26 years, 27–34 years, 35–64 years, and 65 years or older. Usual source of care was categorized as having none, physician’s office, hospital, or emergency room. Urbanicity was categorized as either residing within or outside a Metropolitan Statistical Area. Health and mental health status and were categorized as excellent, very good, good, fair, or poor.

* Measures of CRC Screening
Data on the use of CRC screening originated from several items in the MEPS, including questions asking when the “most recent sigmoidoscopy or colonoscopy” or FOBT occurred. The times of the most recent sigmoidoscopy or colonoscopy and/or FOBT were detailed by the survey as 1 year ago or less, between 1 and 2 years ago, 2 to 3 years ago, between 3 and 5 years ago, more than 5 years ago, more than 10 years ago 
(2009-2012), or never. Table 13 summarizes questions used in the MEPS to assess CRC screening. Table 13. CRC screening questions from the MEPS Screening Medicare population Survey questions Colorectal cancer Men and women 6575 2007-2008 2009-2012 Fecal occult blood test When was the last blood stool test? When was the last blood stool test? Colonoscopy When was the last colonoscopy or sigmoidoscopy? When was the last colonoscopy? Sigmoidoscopy When was the last sigmoidoscopy? Source: Medical Expenditure Panel Survey (MEPS), 2015

Screening for colorectal cancer is accomplished using one or a combination of tests varying in specificity, sensitivity, and risk. The USPSTF recommends screening for colorectal cancer using fecal occult blood testing, sigmoidoscopy, or colonoscopy. CRC screening was defıned as endoscopy (either sigmoidoscopy or colonoscopy) within 5 years and/or FOBT within 1 year as has been described previously (O'Malley, Forrest, Feng, \& Mandelblatt, 2005). The MEPS did not distinguish between sigmoidoscopy and colonoscopy in either 2007 or 2008. To analyze whether individuals were current for CRC screening, three variables were created to describe whether one was current for CRC screening overall (using either FOBT or sigmoidoscopy/colonoscopy), current for CRC screening using FOBT, and current for CRC screening using sigmoidoscopy/colonoscopy. Current overall CRC screening was defined as FOBT within 1 year and/or sigmoidoscopy/colonoscopy within 5 years. Current CRC screening using FOBT was defined as FOBT screening within 1 year. Current CRC screening using sigmoidoscopy/colonoscopy was defined as having had a sigmoidoscopy or colonoscopy procedure within the past 5 years.

Given that colonoscopy is used as a follow up for abnormal FOBT findings, it was possible for an individual to be current for CRC screening using both FOBT and sigmoidoscopy/colonoscopy. In these cases, the individual was coded as current for CRC screening using FOBT to reflect their initial approach to CRC screening. Figure 3 illustrates the relationship between the dependent variables and provides the associated sample size for each.

Covariates (age, sex, residence location, household income, education, usual source of care, supplemental insurance, and health status) were also included. Race/ethnicity was categorized as White non-Hispanic, African American non-Hispanic, and Hispanic and followed previous literature on race/ethnicity disparities among the Medicare population (Doubeni 2012; Shih 2009; Fenton 2009). Categorical variables for four poverty groups were assigned corresponding to poor, low, middle, and high income according to the Federal Poverty Level (FPL):
Marital status was categorized as never married, married, and divorced/separated/widowed. Education was categorized as less than high school, high school, some college, and college graduate. Primary language, based on interview language, was categorized as English or other. Age categories were 65-70 and 71-75 years of age. Usual source of care was categorized as having none, physician’s office, or hospital. Urbanicity was categorized as either residing within or outside a Metropolitan Statistical Area (MSA). Health status was categorized as excellent, very good, good, fair, or poor.



\subsection{Kindratt thesis}


Face-to-face PPC. Face-to-face PPC was evaluated using MEPS and HINTS data sources. To measure the qualities of face-to-face PPC, women who reported that they visited a health care provider in the past 12 months were asked how often health care providers: listened to them; showed respect for what they had to say; spent enough time with them; explained so they could understand; gave them a chance to ask all their questions; gave the

16 the attention needed for their feelings and emotions; involved them in health care decisions; gave them specific instructions (yes or no); confirmed their understanding; and helped them deal with feelings of uncertainty. Participants rated each quality of face-to-face PPC on a four-point scale (never, sometimes, usually or always). A dichotomous variable was created to compare providers who “always” versus “not always” (usually, sometimes or other) exhibited each quality of face-to-face PPC. This method has been used in previous studies evaluating the qualities of face-to-face PPC using MEPS (21, 51-53) and HINTS (54-58) public-use data sources and accounts for the skewed distribution of the data.

Breast cancer screening:
e dependent variable of interest in this study was receipt of mammogram in the last 12 months. The MEPS and HINTS asked women two questions to determine whether or not they had ever received a mammogram and if yes, how long ago they had their last mammogram. A dichotomous variable was created which included women who ever received a mammogram and reported receiving it in the last 12 months compared to those who received it prior to the past year or never received one.
Age, sex, race/ethnicity (non-Hispanic white, Hispanic, non-Hispanic black, or non-Hispanic other race), nativity status (born in the US or not born in the US), marital status (never married, married/living as married or divorced/widowed/separated), highest level of education achieved (no degree, HS degree/GED, some college/Associates degree, or Bachelor’s degree or higher), insurance coverage (insured or uninsured), and perceived health status (excellent/very good/good or fair/poor). Multiple age categories (40-44 years, 45-49 years, 50-54 years, 55-59 years, 60-69 years, or 70 years and older) were created to account for inconsistent recommendations from the ACOG (41), ACS (7) and USPSTF (6).

Pop smear: MEPS asked women two questions to determine whether or not women ever received a Pap test and if yes, how long ago they had their last Pap test (17, 59).

Colon Cancer:
The MEPS asked two questions to determine whether or not adults had ever received a blood stool test, sigmoidoscopy and/or colonoscopy.

\subsection{Tossef}

Our interest centers on measuring the impact of the ACA's broader Medicare coverage for preventive services on the receipt of the eight services listed above, and we examine these changes within four distinct groups of Medicare beneficiaries: (1) beneficiaries with traditional Medicare and no other insurance, (2) beneficiaries with traditional Medicare plus supplemental private insurance (e.g., a Medigap policy), (3) beneficiaries in Medicare managed care plans without other insurance, and (4) beneficiaries with Medicare plus other public coverage (e.g., Medicaid). We expect the first group, beneficiaries with only traditional Medicare, to have benefited the most from the ACA's reforms, because the reforms reduced their copays to zero. The other three groups are included for comparison purposes, as the reforms likely had little or no effect on their copays, which were already zero or near-zero before these reforms (Jensen et al., 2015). Finally, since we want to compare the utilization of preventive services before and after the reforms, we define a binary indicator variable that equals 0 for years 2006–2010 and 1 for years 2012–2016. We exclude 2011 because in the MEPS 2011 data, measurement of utilization of services spills over into part of 2010. 2.5. Covariates We also adjust for several predisposing and enabling factors known to influence the utilization of preventive services (Andersen, 1995). In line with previous work (Jensen et al., 2015), we account for the beneficiary's age (65–69, 70–74, 75–79, 80–84, or 85+), gender (male or female), poverty status based on household income relative to the poverty threshold (poor, near poor, low income, middle income, or high income), marital status (not married or married), educational attainment (less than high school, high school, some college, or college-ormore), and region of residence (Northeast, Midwest, South, or West). We also account for self-reported access to a usual source of care (USC) (yes or no), and two self-reported measures of physical and mental health (excellent, very good, good, fair, or poor). In addition, we control for the individual's health behaviors by including measures related to his or her weekly physical exercise, smoking status, whether their doctor advised exercising more or reducing fatty food intake, the number of times they went to a doctor's office or clinic to receive care, and how long it has been since their last routine check-up by a doctor (within the past year or not). We also control for the individual's attitudes towards health insurance and risk taking, by including measures
of whether they agreed or disagreed with each of the following statements (when considered one at a time): “I am healthy enough that I really don't need health insurance,”“Health insurance is not worth the money it costs,”“I am more likely to take risks than the average person,” and “I can overcome illness without help from a medically trained person.” We also adjust for (1) role of and (2) quality of providers by including variables that reflect the level of “care coordination” and “patient centered care” received by an individual through their USC provider. Specifically, we look at four aspects of “care coordination”: whether they go to their USC provider for: 1) new health problems, 2) preventive health care, 3) referrals to other health professionals, and 4) ongoing health problems. Similarly, we examine “patient centered care” with the USC provider in four ways: does their USC provider 1) ask about prescription medications and treatments other doctors may have given them?, 2) ask about and show respect for medical, traditional and alternative treatments that they are happy with?, 3) ask them to help make decisions between a choice of treatments?, and 4) present and explain all their options to them? We use these eight aspects of their care to derive two latent factors (“care coordination” and “patient centered care”) that capture information contained in the provider quality variables. (All eight variables used to derive the latent factors were assigned the value of 0 if the individual did not have a USC provider or disagreed with the statements). Finally, we control for the individual's total annual number of hospital outpatient department visits.


\subsection{Gawron}

Measures of health status included routinely assessed diseases (diabetes mellitus, asthma, emphysema, hypertension, coronary artery disease, and stroke) and the physical and mental components of the Short Form 12. The number of the different illnesses mentioned was collapsed into the comorbidity burden. Considering the focus on colon cancer screening, we abstracted information about general physical examination, cholesterol, PSA, PAP smear, and mammogram, obtained within the last year as recorded by MEPS. The recorded number of healthcare encounters in an office-based environment was used a measure of healthcare utilization. We finally examined provider characteristics. This domain included questions about interactions with care providers during the year, the ability to reach the provider by phone, to get off-hour appointment, or to receive after-hour care, about provider behaviors (shared decision making and explanation of findings or treatments), and a perceived inability of providers to order or deliver the appropriate care.

o identify independent predictors of access to CRC screening, we first combined fecal occult blood and endoscopic screening methods completed within the last 5 years into a single endpoint (CRC screening). We then performed a binary logistic regression analyses, entering variables that differed significantly between the cohorts with and without screening. We separately coded screening modalities as stool-based testing only, endoscopic testing only, or a combination of the two, looking for independent predictors using a multinomial logistic regression. Finally, we entered the very same variables into binary logistic regressions examining their role as predictors for annual physicals, cholesterol checks, and the sex-specific screening tests, mentioned above, obviously restricting the analysis to the relevant sex. A P < 0.05 was considered as significant difference. Probabilities above this threshold were labeled as nonsignificant (n.s.).

\subsubsection{preventable ED}

\subsection{Wang paper}
Our primary outcome variable was dichotomized, indicating whether a respondent had any preventable ED visits within the past year. We identified those visits using the Prevention Quality Indicators (PQIs) Version 4.5 provided by AHRQ. Each PQI is a set of ICD-9-CM (International Classification of Disease, 9th Revision, Clinical Modification) diagnosis codes—medical conditions that can be managed well by ambulatory care in outpatient settings. Following a previous study [25], 11 out of 14 AHRQ PQIs were used in this study, including 3 acute ACSCs (bacterial pneumonia, dehydration, urinary tract infection) and 8 chronic ACSCs (heart failure, hypertension, angina, asthma between 18 and 39 years old, chronic obstructive pulmonary disease or asthma older than 40 years old, short-term diabetes complications, long-term diabetes complications, and uncontrollable diabetes). The remaining three PQIs, including low birthweight, perforated appendix and diabetes-related lowextremity amputation, were excluded for different reasons. First two conditions (low birthweight and perforated appendix) were only based on pregnancy or on patients with appendicitis, rather than the general adult population. If we had restricted our sample to those conditions, we would have encountered a small analytical sample. In addition, diabetesrelated low extremity amputation only happened in inpatient settings, rather than in the ED [25]. Although the AHRQ PQI measures use fivedigit ICD-9-CM diagnosis codes, our MEPS data only provide the first three digits of each diagnosis code. However, following a prior study, we believethisissufficient to identify preventable ED utilization

The linked MEPS-NHIS data contain respondents' current citizenship and place of birth, which were self-reported and included in the sociodemographic section of the NHIS Family Core data. These data allowed us to classify respondents into three different immigration groups: US native, naturalized citizen, and noncitizen. Respondents born in the United States were defined as US natives. Those who were foreignborn but had already gained US citizenship when interviewed for MEPS were defined as naturalized citizens. The respondents without US citizenship were defined as noncitizens.


Other variables:
Other demographic characteristics and socioeconomic status in this study included age (18–29, 30–44, 45–64, and 65 years and older), sex, race/ethnicity (Hispanic, Non-Hispanic White, Non-Hispanic Black, and Non-Hispanic Others), marital status (married vs. non-married), education level (less than high school, high school, and some college and above), employment status (employed vs. unemployed), poverty (self-reported annual family income less than 125\% of Federal Poverty Line vs. 125\% and above), insurance status (any private insurance, public insurance only, and uninsured), number of outpatient provider visits (0, 1, 2, vs. 3 and above), and usual source of care (yes vs. no). MEPS respondents were asked whether there is a particular doctor’soffice, clinic, health center, or other place that they usually go if they are sick or in need of advice about health. Based on answers to this question, we defined usual source of care as a dichotomized variable. We also used diagnosis codes from MEPS emergency department visits files, officebased medical provider visits files, outpatient files, and hospital inpatient stays files to construct a Charlson Co-Morbidity Index score (0, 1, 2, vs 3 and above) to represent respondents' health needs, with higher values indicating a higher co-morbidity burden. This index is a widely used measure for comorbidities, and it has 17 groups of medical conditions defined by ICD-9-CM diagnosis codes [26,27]. Each condition is given a different weight score ranging from 1 to 6 based on its severity. For non-US born individuals, we included another two immigration-related variables: length of stay in US (less than 5 years vs. 5 years and above), and country of birth (Mexico, Central America, Caribbean Islands; South America; Europe; Asia; Elsewhere).

\subsubsection{Preventable Hospital}

\subsection{Ratakonda pape 2023}
Outcomes Using ICD-9 and ICD-10 codes, we used inpatient claims to identify any occurrence of PPH.3,4,16 In accordance with the definition of the agency for healthcare research and quality, additional exclusion criteria were also considered to ensure an appropriate rate calculation (Supplementary Table 2, available online at http://www.mcpiqojournal.org). The patientyear composite PPH was estimated based on evidence of any PPH occurring each year.

\subsection{Tossef 2019}
For each hospitalization in 2003–2012, MEPS reports up to four ICD-9 codes, each recorded at the 3-digit level. These ICD-9 codes are recorded in the order they were reported by a respondent, not necessarily in their order of clinical importance (AHRQ, 2014a). We generated a binary outcome variable (1 = preventable, 0 = otherwise), avoid_hosp1234, which identifies whether a hospitalization was potentially preventable. Supplementary Table 1 lists the specific Ambulatory Care Sensitive (ACS) conditions that trigger indication of a preventable stay.
\subsubsection{other indeoendent:}
2.2. Independent variables
The key independent variable is a binary variable, mcd_hmo, which equals 1 if the recipient is enrolled in a Medicaid HMO, 0 otherwise. The MEPS includes a multi-step careful ascertainment process to ensure participant enrollment in a Medicaid HMO. Specifically, if Medicaid or other government program was identified as one of the respondent's sources of hospital/physician insurance coverage, he/she was then asked to identify their plan from a list of state names or programs for the Medicaid HMOs in the respondent's area. If the respondent didn't know their plan's name, they were given the following definition of an HMO and asked whether it describes their Medicaid plan: “With an HMO, you must generally receive care from HMO physicians. If another doctor is seen, the expense is not covered unless you were referred by the HMO, or there was a medical emergency.”

Our estimated models account for other factors that could also have influenced occurrence of a preventable hospitalization, including demographics, health and functional status, attitudes towards health insurance and risk-taking, and use of preventive services. These covariates have been previously adopted in studies examining preventable hospitalizations and emergency department utilization (Culler et al., 1998). Demographics include age (<35, 35–55, and 56 and above), gender, poverty status based on household income relative to poverty thresholds (poor, near poor, low income, and middle-or-high income), education (high school or less, some college, and college or more), and region (northeast, midwest, south, and west).

Health and functional status measures include self-reported health (good/very good/excellent and fair-or-poor), self-reported mental health (good/very good/excellent, and fair-or-poor), whether he/she has any difficulty with activities of daily living (ADLs), whether he/she has any difficulty with instrumental activities of daily living (IADLs), adult Body Mass Index (underweight, normal, overweight, obese), whether he/she has been advised to restrict fatty foods, whether he/she currently smokes, and whether he/she has a usual source of care. We also account for the self-reported presence/absence of ten clinical conditions, each measured by a (0,1) indicator, including the presence of high blood pressure, coronary heart disease (CHD), other heart disease, angina, emphysema, diabetes, and asthma, ever having had a heart attack or myocardial infarction, and ever having had a stroke. AHRQ refers to these conditions as priority conditions due to their high prevalence (AHRQ, 2014b).

To control for attitudes towards health insurance and risk-taking we include four variables that measure whether the respondent agrees with each of four statements (considered one at a time): “I'm healthy enough that I really don't need health insurance,” “Health insurance is not worth the money it costs,” “I'm more likely to take risks than the average person,” and “I can overcome illness without help from a medically trained person.”

Preventive services utilization measures include indicators probing the length of time since the respondent's last routine check-up, the length of time since their last cholesterol check, and the length of time since their last flu shot. These variables proxy for how conscientious a person is about taking care of their own health, which may correlate with their ability to recognize potentially dangerous symptoms or when they should see their doctor.

\section{Standared error}
\subsection{Xiaobei thesis }

Standard Error Estimation MEPS-HC is based on a complex stratified multistage area sample design in which the primary sampling unit (PSU) is a county or cluster of counties (AHRQ, 2019b). In addition, its overlapping panel design determines that the same persons show up in two different cross-sections. Hence, there is correlation not only among persons
within the same PSU but also between the two observations for the same person. The major consequence of serial correlation is causing underestimated standard errors (Cameron \& Miller, 2015). To account for correlation at all stages and the complex sampling design, I followed AHRQ’s guidelines that include specifying sampling strata and PSU with the provided variables and applying the provided sample weight in variance estimation (AHRQ, 2019b).

\section{limitation}

\subsection{Bloodworth}
This study has several limitations. In terms of datasets, MEPS uses a cross-sectional design, which makes it impossible to observe longitudinal trends at the individual level. Since MEPS is an interview survey, the data are self-reported, which can lead to recall bias. Additionally, MEPS data were only available through 2014 at the time of this study, so there may not be enough post-ACA data to detect true effects. Lastly, the Kaiser survey of state Medicaid preventive service policies only asked the Medicaid programs whether they required a copay for each service, not the cost of the copay.

\subsection{ Tossef Paper}

Furthermore, the publicly available MEPS data files only contain information on ICD-9 codes up to three-digits, while AHRQ PQI measures use the full five-digit codes. However, prior studies have used three-digits ICD-9 codes for their analysis and we believe this classification is sufficient for our analysis also (Galarraga et al., 2015; Galarraga and Pines, 2016; Wang et al., 2018). In order to further address this concern, we performed a sensitivity analysis in which we excluded all hospital visits for dehydration and performed the same analysis. Then we excluded all hospital visits for diabetes and performed the same analysis. We noticed no significant changes in our findings as compare to the original findings and so reached at consistent conclusions. Furthermore, MEPS relies on self-report to identify the respondents' insurance plans and health services use. This may give rise to measurement error. However, MEPS has been used extensively to study individuals' health usage and expenditure patterns. Secondly, there is evidence that reported insurance status tends to be pretty accurate within the MEPS (Hill, 2007). In addition, there is also published evidence that MEPS respondents accurately report their inpatient hospitalizations (Zuvekas and Olin, 2009) Overall, we believe that MEPS is a reliable data source when it comes to the question in hand. Correcting each of these limitations and using a larger sample size to overcome any power issues represents a fruitful direction for further research on the performance of HMOs under Medicaid.

\end{document}


