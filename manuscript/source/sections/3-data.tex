\documentclass[../main.tex]{subfiles}

\begin{document}

\section{Data Source}

The primary data obtained from the Medical Expenditure Panel Survey Household Component (MEPS-HC), which represents the non-institutionalized US population on a national scale. Each year MEPS-HC selects a new panel of sample households from individuals who participated in the National Health Interview Survey(NHIS) conducted by the National Center for Health Statistics in the previous year.  and conducts interviews with each household for two successive calendar years, involving five rounds of data collection. Using an overlapping panel approach, data from two panels cover each calendar year. This extensive survey gathers information on a broad array of topics, including demographic characteristics, socioeconomic status, health insurance coverage, access to healthcare, health status,  health conditions, healthcare utilization, and health-related expenditures\parencite{ahrq_medical_2024}.
The medical conditions related to each visit were classified by expert coders using the exact text provided by each participant. AHRQ has conducted verification of the codes, confirming that their error rate is below 2.5\% \parencite{wang_fewer_2018}. Once the Household interviews were concluded, respondent medical providers were contacted by telephone in order to get supplementary information regarding medical visit summaries, diagnostic codes, and billing. This Second telephone survey create the Medical Component (MEPS-MC) the access to this component is restricted to use on AHRQ Data Center or FSRDCs \parencite{ahrq_medical_2024}. For this study I compile nine years of cross-sectional data from the household component, covering the period from 2011 to 2019, which includes time frames both before and after the Medicaid expansion. While all data in MEPS-HC are publicly available, the state identifier is an exception. This variable is crucial for constructing the expansion status, that indicates whether a respondent lived in a state that expanded its Medicaid program, and for specifying state fixed effects in the regression models. Access to these identifier variables will be facilitated through an application to AHRQ to utilize the MEPS-AWS Secure Cloud.

MEPS was chosen as the ideal dataset due to its comprehensive array of relevant variables that closely align with the study objectives. and its widespread utilization in examining the effects of the Affordable Care Act (ACA)


\section{variables}

Our original sample contained 113,499 adults aged 18 and older. After excluding respondents with any missing values in covariates (3.4\% of the sample), the final
analytical sample size was 109,602, including 80,911 US natives, 12,809 naturalized citizens, and 15,882 noncitizens. Regarding power and sample size
considerations, there has been some debate (Dziak et al., 2020) on the utility of conducting power and sample size calculations for analysis of data that have already been collected -
as is the case with the MEPS data. Given that context, an exploratory analysis was conducted to determine the specific number of non-Hispanic Black and non-Hispanic White men who contributed at least one observation to the MEPS database between 2011 and 2019 (see Figure 4 and Figure 5 in the appendix)

\end{document}


