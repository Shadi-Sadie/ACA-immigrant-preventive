\documentclass[../main.tex]{subfiles}

\begin{document}

\section{Data Source}

The primary data obtained from the Medical Expenditure Panel Survey Household Component (MEPS-HC), which represents the non-institutionalized US population on a national scale. Each year MEPS-HC selects a new panel of sample households from individuals who participated in the National Health Interview Survey(NHIS) conducted by the National Center for Health Statistics in the previous year.  and conducts interviews with each household for two successive calendar years, involving five rounds of data collection. Using an overlapping panel approach, data from two panels cover each calendar year. This extensive survey gathers information on a broad array of topics, including demographic characteristics, socioeconomic status, health insurance coverage, access to healthcare, health status,  health conditions, healthcare utilization, and health-related expenditures\parencite{ahrq_medical_2024}.
The medical conditions related to each visit were classified by expert coders using the exact text provided by each participant. AHRQ has conducted verification of the codes, confirming that their error rate is below 2.5\% \parencite{wang_fewer_2018}. Once the Household interviews were concluded, respondent medical providers were contacted by telephone in order to get supplementary information regarding medical visit summaries, diagnostic codes, and billing. This Second telephone survey create the Medical Component (MEPS-MC) the access to this component is restricted to use on AHRQ Data Center or FSRDCs \parencite{ahrq_medical_2024}. For this study I compile nine years of cross-sectional data from the household component, covering the period from 2011 to 2019, which includes time frames both before and after the Medicaid expansion. While all data in MEPS-HC are publicly available, the state identifier is an exception. This variable is crucial for constructing the expansion status, that indicates whether a respondent lived in a state that expanded its Medicaid program, and for specifying state fixed effects in the regression models. Access to these identifier variables will be facilitated through an application to AHRQ to utilize the MEPS-AWS Secure Cloud.

MEPS was chosen as the ideal dataset due to its comprehensive array of relevant variables that closely align with the study objectives. and its widespread utilization in examining the effects of the Affordable Care Act (ACA)


\section{Sample}

Our original sample contained 113,499 adults aged 18 and older. After excluding respondents with any missing values in covariates (3.4\% of the sample), the final
analytical sample size was 109,602, including 80,911 US natives, 12,809 naturalized citizens, and 15,882 noncitizens. Regarding power and sample size
considerations, there has been some debate (Dziak et al., 2020) on the utility of conducting power and sample size calculations for analysis of data that have already been collected -
as is the case with the MEPS data. Given that context, an exploratory analysis was conducted to determine the specific number of non-Hispanic Black and non-Hispanic White men who contributed at least one observation to the MEPS database between 2011 and 2019 (see Figure 4 and Figure 5 in the appendix)

\section{Dependent Variables}

The outcome of interest is recipient of the following preventive screenings recommended by USPSTF

Mammograms in the past two years
Sigmoidoscopy in the past five years
Colonoscopy in the past ten years

After 2018, the exact variables for these screenings are readily available due to changes in the survey's preventive questioning. However, for years prior to 2018, I construct these three variables for eligible respondents. Eligibility criteria include being within the recommended age for screening and having no prior diagnosis of breast or colorectal cancer. This construction relies on respondents' answers to questions regarding their most recent sigmoidoscopy, colonoscopy, or mammogram.



\footnote{Before 2018, each panel was surveyed twice regarding preventive screening: once in the first year of the interview (Round 3) and again in the second year (Round 5). However, beginning in 2018, the Agency for Healthcare Research and Quality (AHQR) changed their approach to preventive questioning. They altered the format of questions; for instance, the query regarding colorectal cancer screening changed from "When was your last sigmoidoscopy?" to "Have you had a sigmoidoscopy in the last 5 years?" Additionally, they decided to ask this question only once per panel, either in Round 4 or Round 2. Consequently, preventive screening questions are now visible in alternating years.

As previously mentioned, data from two panels are collected for each calendar year. For example, in 2017, data is gathered from panels 21 and 22, while in 2018, it's from panels 22 and 23. This means that individuals in Panel 22 are present in both 2017 and 2018. since the same individual may contribute to years without preventive care data, I've created a variable based on their responses from another year. For instance, although there's no data on preventive screening in 2017, we know that Panel 21 and Panel 22 contributed to this year's data, and individuals in Panel 22 were queried about their preventive screenings in 2018, while those in Panel 21 were asked in 2016. Therefore, I've constructed a cancer screening variable for 2017 using responses from Panel 22 in 2018 and individuals from Panel 21 included in the 2016 dataset. The three variables I'm reconstructing are: having had a sigmoidoscopy in the last 5 years, having had a colonoscopy in the last 10 years, and having had a mammogram in the last two years.

It's important to note that by doing this, I might be underestimating the number of people in Panel 21 who may have undergone preventive screening in the following year. I will be employing a similar construction for 2019 as well.}


\section{Key Independent Variables}


 control for various predisposing and enabling factors known to impact the utilization of preventive services (Andersen, 1995). Consistent with prior research (Jensen et al., 2015), I include


\end{document}


