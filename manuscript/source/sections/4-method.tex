\documentclass[../main.tex]{subfiles}

\begin{document}




\end{document}



For this study, two primary data sources are utilized to examine the impact of Medicaid expansion on immigrant insurance coverage and disparities in the United States. The American Community Survey (ACS), accessible via the Public Use Microdata Sample(PUMS) serves as our main data source. The ACS, an annual survey, collects comprehensive data on the U.S. population and various aspects of their lives. It offers several advantages for our research: a substantial sample size representing approximately 1\% of the U.S. population (around 3 million individuals yearly), high response rates ranging from 92\% to 98\%, national representatives, and state-level granularity. The ACS includes a wide array of data, encompassing detailed health insurance coverage, income, citizenship status, country of origin, and socio-economic and demographic information. Furthermore, it provides geographic data at the state and Public Use Microdata Area (PUMA) levels, enabling us to track changes in insurance rates and coverage sources over time The downside of this data set is that it does not collect information on immigrants' documentation status, limits us from distinguishing the effects of Medicaid expansion on undocumented and documented immigrants in the U.S.
